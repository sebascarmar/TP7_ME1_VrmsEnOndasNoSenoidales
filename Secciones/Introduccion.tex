  \section{Introducción}
    La mayoria de los multímetros del mercado, cuando se realiza mediciones
    en CA, poseen un conversor de valor medio calibrado para indicar el 
    valor eficaz mediante la aplicación de un \textit{factor de relación}. 
    Por lo general, la relación del factor se hace superponiendo ondas 
    senoidales puras, por eso mismo, cuando se trata de medir otras señales 
    como ondas cuadradas, triangulares, trenes de pulso, etc;
    Se obtiene un error en la lectura, debido a que el factor de relación 
    es propio de cada tipo de señal. En el presente trabajo practico, 
    se realizara la contrastacion entre un multímetro de uso general con uno
    que posee medición al verdadero valor eficaz \textit{TRUE RMS}, con el 
    fin de poder realizar cotas de corrección para señales que no sean 
    senoidales puras. 
       