  \section{Introducción}
    La mayoría de los multímetros del mercado, para realizar mediciones
    en corriente alterna, poseen un conversor de valor medio del módulo 
    calibrado para indicar el valor eficaz mediante la aplicación de 
    un \textit{factor de relación o forma}. Por lo general, la relación del factor 
    se hace superponiendo ondas senoidales puras, por eso mismo, cuando se trata de medir 
    otras señales como ondas cuadradas, triangulares, trenes de pulso, etc.,
    se obtiene un error en la lectura, debido a que el factor de relación 
    es propio de cada tipo de señal. En el presente trabajo práctico, 
    se realizará una comparación entre un multímetro de uso general con uno
    que posee medición del verdadero valor eficaz \textit{TRUE RMS}. 
       