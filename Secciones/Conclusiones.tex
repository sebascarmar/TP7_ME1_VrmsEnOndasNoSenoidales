  \pagebreak
  \section{Conclusiones}
    En relación con el primer experimento realizado en este trabajo
    práctico, se procede a obtener de forma analítica los valores
    de las cotas de corrección de las mediciones realizadas con
    el multímetro con respuesta al valor medio del módulo de la 
    tensión.
    
    Se sabe que los multímetros de respuesta al valor medio del
    módulo realizan una corrección mediante una cota para obtener 
    el valor eficaz de la medición. Para obtener dicha cota, se 
    hace uso de los siguientes valores representativos de una 
    \textbf{señal senoidal}

    \vspace{-5pt}
    $$ V_{RMS_{\sin}} = \dfrac{V_{\max}}{\sqrt{2}} \hspace{20pt} ; 
    \hspace{20pt} V_{|med|_{\sin}} = \dfrac{2\, V_{\max}}{\pi}~. $$

    \noindent Luego, la cota de corrección (factor de forma) se obtiene
    relacionando las expresiones anteriores
    
    \vspace{-5pt}
     $$ k = \dfrac{V_{RMS_{\sin}}}{V_{|med|_{\sin}}} 
        = \dfrac{\dfrac{V_{\max}}{\sqrt{2}}}{\dfrac{2\, V_{\max}}{\pi}}
        \hspace{20pt} \Longrightarrow \hspace{20pt} k = 1,1107~.$$

    El valor antes encontrado, es el que permite saber que el multímetro
    de respuesta al valor medio del módulo muestra en el display el 
    resultado de $ V_{lectura} = 1,1102 \cdot V_{|med|}~. $

    Teniendo en cuenta lo mencionado anteriormente, y que para una señal cuadrada
    su valor eficaz y valor medio de módulo se expresan mediante

    \vspace{-5pt}
    $$ V_{RMS_{cuad}} = V_{\max} \hspace{20pt} ; \hspace{20pt} V_{|med|_{cuad}} = V_{\max}~, $$

    \noindent entonces, la cota de corrección para la medición de 
    la señal cuadrada es

    \vspace{-5pt}
    $$ e_{cuad} [\%] = \dfrac{V_{RMS_{cuad}} - k\cdot V_{|med|_{cuad}}}{k \cdot V_{|med|_{cuad}}} \cdot 100
              = \dfrac{V_{\max} - 1,1107\, V_{\max}}{1,1107\, V_{\max}} \cdot 100 $$
              $$  \therefore \hspace{20pt} \boxed{e_{cuad}[\%] = -9,97\%}~.
    $$

    Siguiendo la misma línea, para una señal triangular sus valor eficaz y valor
    medio de módulo se expresan mediante

    \vspace{-5pt}
$$ V_{RMS_{tri}} = \dfrac{V_{\max}}{\sqrt{3}} \hspace{20pt} ; \hspace{20pt} V_{|med|_{tri}} = \dfrac{V_{\max}}{2}~, $$

    \noindent entonces, la cota de corrección para la medición de 
    la señal triangular es

    \vspace{-5pt}
    $$ e_{tri} [\%] = \dfrac{V_{RMS_{tri}} - k\cdot V_{|med|_{tri}}}{k \cdot V_{|med|_{tri}}} \cdot 100
    = \dfrac{ \dfrac{V_{\max}}{\sqrt{3}} - 1,1107\, \dfrac{V_{\max}}{2}}{1,1107\, \dfrac{V_{\max}}{2}} \cdot 100 $$
              $$  \therefore \hspace{20pt} \boxed{e_{tri}[\%] = +3,96\%}~.
    $$

    En la Tabla~\ref{tab:ComparacionCotas} se puede observar la comparación entre valores
    prácticos y teóricos de las mediciones realizadas con el multímetro de respuesta al 
    valor medio del módulo. Los valores presentan diferencias entre sí, en especial
    en el caso de la señal triangular. Esto se puede explicar en base a que el
    instrumento no está diseñado para medir este tipo de señales de forma correcta. 

    \begin{table}[H] \centering
      \begin{tabular}{|c|c|c|c|} \hline
        \textbf{Señal}       & \textbf{Error práctico [\%]}  & \textbf{Error teórico [\%]} & \textbf{Factor de forma} \\ \hline
        Cuadrada    & -10,84               & -9,97              &  1              \\ \hline
        Triangular  & +2,18                & +3,96              &  1,1547         \\ \hline
      \end{tabular}
      \caption{Tabla de comparación de cotas.}
      \label{tab:ComparacionCotas}
    \end{table}









