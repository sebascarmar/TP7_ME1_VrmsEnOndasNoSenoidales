\subsection{Valor medio}
    EL valor medio de una señal, viene dado por el 
    \textit{teorema de la media} el cual indica que, si una función \(i(t)\)
    es continua en un intervalo \([a,b]\), existirá un punto \(\eta\) 
    tal que
    \begin{equation}
        \int_{a}^{b} i(t)~dt = (b-a)\cdot i(\eta)~. \label{eqn:EcuMedia}
    \end{equation}

    \noindent Ahora si el intervalo presentado, es igual a un período T, entonces 
    el valor \(i(\eta)\) se concidera el valor medio de una señal \(i(t)\), 
    \(I_{med} = i(\eta)\). Despejando el valor medio de la ecuación~\ref{eqn:EcuMedia}
    \begin{equation*}
        I_{med}= \dfrac{1}{T}\int_{0}^{T} i(t)~dt~.
    \end{equation*}

    El valor de la integral, puede ser cero en caso de que la señal sea simétrica, es decir,
    que su área positiva sea igual que la negativa, como en señales seoidales puras o 
    cuadradas.
    
    \noindent Por ende, en caso de señales que posean un valor medio nulo, se calcula el
    \textit{valor medio del módulo} tomando la integral a lo largo de un período 
    |\(i(t)\)| de la señal
    \begin{equation*}
        I_{|med|} ~ = \dfrac{1}{T} \int_{0}^{T} |i(t)|~dt~.
    \end{equation*}

        \subsubsection{Obtencion del valor medio en una señal Seoidal}

            Debido a que el valor medio de una onda senoidal es cero, se utiliza el 
            valor medio del módulo de la misma. Por ende partiendo de una onda senoidal
            con valor medio de modulo \(V_p\sin(\omega t)\), integrado en un periodo \(\pi\)
            \begin{equation*}
                V_{|med|} ~ = \dfrac{1}{\pi} \int_{0}^{\pi} |V_p~\sin(\omega t)|~dt~.
            \end{equation*}

            \noindent Resolviendo la integral, se obtiene el valor medio del modulo de una 
            onda senoidal 
            \begin{equation*}
                V_{|med|} = \dfrac{2~V_p}{\pi}~.
            \end{equation*}
            
            \noindent Por ultimo, como en el presente trabajo practico se 
            utilizará señales cuadráticas y triangulares los \(V_{|med|}\)
            de dichas señales son
            \begin{equation*}
                V_{|med|_{cuad}} = V_p \hspace{20pt};\hspace{20pt} 
                V_{|med|_{triang}} = \dfrac{V_p}{2}~.
            \end{equation*}
