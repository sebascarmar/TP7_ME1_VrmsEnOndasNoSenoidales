  \subsection{Medición de tensión eficaz de ondas no sinusoidales}
    Haciendo uso del generador de funciones se setean señales de forma cuadrada y
    triangular de una amplitud de $5~V_{pp}$ y frecuencia de $50~Hz$. Luego, se 
    mide el valor de tensión con ambos multímetros, se calcula el error, y finalmente,
    se tabula.

    \subsubsection{En una onda cuadrada}
      En la Figura~\ref{fig:SeñalCuadrada} se puede observar la señal cuadrada seteada.
      Luego, en la Figura~\ref{fig:MedicionSeñalCuadrada} se observa la medición de tensión
      con ambos multímetros.

      \begin{figure}[H]
        \centering
        \frame{\includegraphics[width=0.5\textwidth]{Imagenes/ActividadPractica/MedicionDeTensionEficaz/Exp1_SeñalCuadrada.jpeg}}
        \caption{Señal cuadrada a medir.}
        \label{fig:SeñalCuadrada}
      \end{figure}

      \begin{figure}[H]
        \centering
        \frame{\includegraphics[width=0.5\textwidth]{Imagenes/ActividadPractica/MedicionDeTensionEficaz/Exp1_MedicionSeñalCuadrada.jpeg}}
        \caption{Mediciones de la señal cuadrada.}
        \label{fig:MedicionSeñalCuadrada}
      \end{figure}

      Luego, la cota de correción para el multímetro de respuesta al valor medio del módulo es

      \begin{align*}
        e [\%] = \dfrac{V_{RMS} - V_{|med|}}{V_{|med|}} \cdot 100
               = \dfrac{2,4~V - 2,2~V}{2,2V} \cdot 100
               \hspace{20pt} \Longrightarrow \hspace{20pt} \Aboxed{e = +9,09\%}~.
      \end{align*}


    \subsubsection{En una onda triangular}
      En la Figura~\ref{fig:SeñalTriangular} se puede observar la señal cuadrada seteada.
      Luego, en la Figura~\ref{fig:MedicionSeñalTriangular} se observa la medición de tensión
      con ambos multímetros.

      \begin{figure}[H]
        \centering
        \frame{\includegraphics[width=0.5\textwidth]{Imagenes/ActividadPractica/MedicionDeTensionEficaz/Exp1_SeñalTriangular.jpeg}}
        \caption{Señal triangular a medir.}
        \label{fig:SeñalTriangular}
      \end{figure}

      \begin{figure}[H]
        \centering
        \frame{\includegraphics[width=0.5\textwidth]{Imagenes/ActividadPractica/MedicionDeTensionEficaz/Exp1_MedicionSeñalTriangular.jpeg}}
        \caption{Mediciones de la señal triangular.}
        \label{fig:MedicionSeñalTriangular}
      \end{figure}

      Luego, la cota de correción para el multímetro de respuesta al 
      valor medio del módulo es

      \begin{align*}
        e [\%] = \dfrac{V_{RMS} - V_{|med|}}{V_{|med|}} \cdot 100
               = \dfrac{1,4~V - 1,0~V}{1~V} \cdot 100
               \hspace{20pt} \Longrightarrow \hspace{20pt} \Aboxed{e = +40\%}~.
      \end{align*}


  
