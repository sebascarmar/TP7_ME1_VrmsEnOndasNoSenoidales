\subsection{Valor eficaz}

    El valor eficaz es aquel que produce la misma disipación de potencia 
    media (valor medio de la potencia instantánea) sobre un 
    resistor, que la que disipa una señal de continua.
    
    Partiendo de una corriente \(i(t)\) en un periodo T, que circula 
    por una resistencia R, disipando una potencia instantánea  \(P_a = i^2(t)~R\).
    Por ende, la potencia media será
    \begin{equation*}
        P_a = \dfrac{1}{T} \int_{0}^{T} P_a(t)~dt ~ = ~ 
        \dfrac{1}{T} \int_{0}^{T} i^2(t)~R~dt ~.
    \end{equation*}

    Ahora, suponiendo una potencia instantánea \(P_c = I^2(t)~R\) debido a una 
    corriente continua sobre la misma resistencia R,
    al igualar \(P_a = P_c\), se obtiene 
    \begin{equation*}
        \dfrac{1}{T} \int_{0}^{T} i^2(t)~R~dt = I^2(t)~R~.    
    \end{equation*}

    Por último, se ve que la corriente que produce la
    misma disipación de potencia que la señal de alterna es 
    \begin{equation}
        I_{ef} = \sqrt{\dfrac{1}{T} \int_{0}^{T} i^2(t)~dt}~. \label{eqn:Veficaz}
    \end{equation}

    \noindent El valor eficaz obtenido, es la raíz cuadrática media
    de la señal \(i(t)\), conocido por sus siglas en inglés como 
    \textit{root mean square} (RMS).

        \subsubsection{Obtención del valor eficaz en una señal senoidal}

            Partiendo de una señal \(V_p\sin(\omega t)\) y de la ecuación (\ref{eqn:Veficaz}), 
            integrada en un período \(2\pi\)
            \begin{equation*}
                V_{ef} = \sqrt{\dfrac{1}{2\pi} \int_{0}^{2\pi} (V_p~\sin(\omega t))^2~dt}~,        
            \end{equation*}

            \noindent al resolverla se obtiene     
            \begin{equation*}
                V_{ef} = \dfrac{V_p}{\sqrt{2}}~.    
            \end{equation*}    

            Para señales cuadradas y triangulares el \(V_{ef}\) es
            \begin{equation}
                V_{ef_{cuadr}}  =  V_p \hspace{20pt} ; \hspace{20pt}    \label{eqn:VeficazdeSeñales}
                V_{ef_{triang}} = \dfrac{V_p}{\sqrt{3}}~.
            \end{equation}