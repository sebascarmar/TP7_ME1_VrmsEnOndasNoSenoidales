\subsection{Valor eficaz}

    El valor eficaz, es aquel que produce la misma disipación de potencia 
    media, es decir valor medio de la potencia instantánea, sobre un 
    resistor que una señal de continua.
    
    \noindent Supongamos una corriente \(i(t)\) en un periodo T, la cual 
    al circular por una resistencia R, disipando una potencia instantánea 
    \begin{equation*}
        P_a = i^2(t)~R~.
    \end{equation*}

    \noindent Por ende la potencia media será
    \begin{equation*}
        P_a = \dfrac{1}{T} \int_{0}^{T} P_a(t)~dt ~ = ~ 
        \dfrac{1}{T} \int_{0}^{T} i^2(t)~R~dt ~.
    \end{equation*}

    \noindent Ahora, suponiendo una potencia instantánea debido a una 
    corriente contiuna sobre la misma resistencia R
    \begin{equation*}
        P_c = I^2(t)~R~.
    \end{equation*} 

    \noindent Si se iguala \(P_a = P_c\), se obtiene 
    \begin{equation*}
        \dfrac{1}{T} \int_{0}^{T} i^2(t)~R~dt = I^2(t)~R~.    
    \end{equation*}

    \noindent Por ultimo, vemos que la corriente que podruce la
    misma disipación de potencia que la señal de alterna es 
    \begin{equation*}
        I_{ef} = \sqrt{\dfrac{1}{T} \int_{0}^{T} i^2(t)~dt}~.
    \end{equation*}

    \noindent El valor eficaz obtenido, es la raiz cuadratica media
    dela señal \(i(t)\), conocido por sus siglas en ingles como RMS 
    (\textit{root mean square}) 

        \subsubsection{Obtencion del valor Eficaz en una señal Seoidal}

            Partiendo de una señal \(V_p\sin(\omega t)\), integrada en un 
            periodo \(2\pi\)
            \begin{equation*}
                V_{ef} = \sqrt{\dfrac{1}{2\pi} \int_{0}^{2\pi} (V_p~\sin(\omega t))^2~dt}~.        
            \end{equation*}
            
            \noindent Resolviendo la ecuacion, se obtiene el \(V_{ef}\) 
            para una onda seoidal 
            \begin{equation*}
                V_{ef} = \dfrac{V_p}{\sqrt{2}}~.    
            \end{equation*}    

            \noindent Para señales Cuadradas y triangulaes el \(V_{ef}\) será
            \begin{equation*}
                V_{ef_{cuadr}}  =  V_p \hspace{20pt} ; \hspace{20pt}
                V_{ef_{triang}} = \dfrac{V_p}{\sqrt{3}}~.
            \end{equation*}